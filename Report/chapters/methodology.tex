\chapter{Methodology} \label{chap:methodology}
This chapter lays out the methodical steps I took while implementing the project.
Main steps for implementing this project fall into four general categories.
These four categories are:
\begin{itemize}
    \item Establishing a baseline benchmark
    \item Improving metrics beyond baseline benchmark
    \item Ensuring model interpretability is maintained
    \item Deploying the final model to production
\end{itemize}

Further sections will provide detail for these four main categories.

\section{Establishing a benchmark}
The first step to every machine learning project is that model produced is to perform better than any random or naive solution.
Although random guess in any binary classification suggests that the minimum accuracy should be greater or equal than 50\%, because of the imbalance in the test dataset actually requires higher accuracy for this project.
Mainly because we have 390 pneumonia images in the 624 total images in the test dataset dummy classifier that predict pneumonia for any given image will achieve 62.5\% accuracy.
However, given the imbalance of the test dataset accuracy would not be the choice of a good metric for assessing the performance.
For that reason on any experiment conducted, collecting precision, recall as well as calculating f1 score will be utilized.
If I introduce these metrics in more detail.
Precision is calculated by dividing true positives (tp) in predictions to true positives and false positives (fp).
\begin{equation}
    Precision = \frac{tp}{tp + fp}
\end{equation}

The recall is calculated by dividing true positives to true positives and false negatives (fn).

\begin{equation}
    Recall = \frac{tp}{tp + fn}
\end{equation}

To capture the correct performance of the classifier we need to consider both precision and recall.
F1 score provides the ability to consider both precision and recall for the same classification problem because it is calculated by getting a harmonic mean of the precision and recall.

\begin{equation}
    F_1 = \frac{2}{\frac{1}{precision} + \frac{1}{recall}} = 2 \times \frac{precision \times recall}{precision + recall}
\end{equation}


Next part for the benchmarking is choosing which algorithms to train on the data for benchmarking.
Given we don't know the distribution of the data, training fundamental machine learning algorithms together with the neural network algorithms is a cautious step to take. 
For that reason, I have chosen to train two fundamental machine learning algorithm, namely the Random Forest classifier and the Support vector classifier as a part of establishing a benchmark.


\subsection{Random Forest Classifier}
For the past years, tree-based algorithms have been very popular in the academic community as well as in the industry with numerous papers demonstrated in ICML, NIPS and JMLR.
Random forest emerges in this category as a very strong algorithm by Breiman~\cite{randomforest} that achieves remarkable performance in small to a medium dataset.
The final baseline will be determined by finding an optimal number of estimators and maximum features using cross-validation.


\subsection{SVM Classifier}
In addition to Random Forest, the secondary mainstream machine learning algorithm is Support Victor Machine classifier~\cite{svm}.
The biggest factor for choosing this algorithm was its robustness in detecting non-linear features in the data using kernel trick.
Similarly to Random Forest hyper-parameters chosen with cross-validation.

\subsection{LeNet-5}
LeNet-5~\cite{Lenet5} is the first one of the well-known CNNs that I will apply to this problem to set a baseline benchmark.
I kept the attributes of the network as close to the origin network as possible but some characteristic of the network had to be changed when it is trained on this classification problem.
The first difference in this project needed in input and output layers, MNIST dataset have ten classes therefore required ten dense neurons at the output layer in the original network but in this project, there are two classes to predict which can be achieved by having one dense output neuron instead.
Input layer also changed because $32 \times 32$ is not a suitable resolution for the pneumonia detection and changed to $224 \times 224$ for better representation.
Another change made to this model is using relu activation function rather than hyperbolic tangent function (tanh) because of the less than ideal gradient flow in saturated tanh function prevented model to converge to a good solution.
Detail of the model summary given below for reference.

\begin{verbatim}
    Model: "LeNet-5"
_________________________________________________________________
Layer (type)                 Output Shape              Param #   
=================================================================
conv2d_6 (Conv2D)            (None, 220, 220, 6)       456       
_________________________________________________________________
average_pooling2d_4 (Average (None, 110, 110, 6)       0         
_________________________________________________________________
conv2d_7 (Conv2D)            (None, 106, 106, 16)      2416      
_________________________________________________________________
average_pooling2d_5 (Average (None, 53, 53, 16)        0         
_________________________________________________________________
conv2d_8 (Conv2D)            (None, 49, 49, 120)       48120     
_________________________________________________________________
flatten_2 (Flatten)          (None, 288120)            0         
_________________________________________________________________
dense_6 (Dense)              (None, 120)               34574520  
_________________________________________________________________
dense_7 (Dense)              (None, 84)                10164     
_________________________________________________________________
dense_8 (Dense)              (None, 1)                 85        
=================================================================
Total params: 34,635,761
Trainable params: 34,635,761
Non-trainable params: 0
_________________________________________________________________
\end{verbatim}

\subsection{AlexNet}
Similar to LeNet-5, AlexNet~\cite{Alexnet} also aimed to the kept original structure as much as possible.
AlexNet comprises of eight layers, five of those were convolution layers where some of them connected to max-pooling layer. 
Despite the fact, the architect is preserved almost same as the original implementation additional steps such as adding local response normalization or PCA augmentation is not applied because of the ad hoc nature of the process and limited effect of this processes in the final performance.
AlexNet is ultimately a very influential paper that steers the direction for how the CNNs are designed and fueled the adoption of the use of neural networks in many fields.
Fallowing quote from the paper also explains the reason why neural networks gain so much popularity and pushing the state of the art results year after year.
\begin{quote}
    \textit{"All of our experiments suggest that our results
can be improved simply by waiting for faster GPUs and bigger datasets to become available."}
\end{quote}

\begin{verbatim}
    Model: "AlexNet"
_________________________________________________________________
Layer (type)                 Output Shape              Param #   
=================================================================
conv2d_10 (Conv2D)           (None, 54, 54, 96)        34944     
_________________________________________________________________
max_pooling2d_6 (MaxPooling2 (None, 26, 26, 96)        0         
_________________________________________________________________
conv2d_11 (Conv2D)           (None, 26, 26, 256)       614656    
_________________________________________________________________
max_pooling2d_7 (MaxPooling2 (None, 12, 12, 256)       0         
_________________________________________________________________
conv2d_12 (Conv2D)           (None, 12, 12, 384)       885120    
_________________________________________________________________
conv2d_13 (Conv2D)           (None, 12, 12, 384)       1327488   
_________________________________________________________________
conv2d_14 (Conv2D)           (None, 12, 12, 256)       884992    
_________________________________________________________________
max_pooling2d_8 (MaxPooling2 (None, 5, 5, 256)         0         
_________________________________________________________________
flatten_2 (Flatten)          (None, 6400)              0         
_________________________________________________________________
dense_6 (Dense)              (None, 4096)              26218496  
_________________________________________________________________
dropout_4 (Dropout)          (None, 4096)              0         
_________________________________________________________________
dense_7 (Dense)              (None, 4096)              16781312  
_________________________________________________________________
dropout_5 (Dropout)          (None, 4096)              0         
_________________________________________________________________
dense_8 (Dense)              (None, 1)                 4097      
=================================================================
Total params: 46,751,105
Trainable params: 46,751,105
Non-trainable params: 0
_________________________________________________________________
\end{verbatim}

\subsection{VGGNet}

\subsection{ResNet}

\section{Improving Performance}
\subsection{Ensambling Models}
\subsection{Transfer Learning}

\section{Model Interpretability}
\subsection{GradCAM}

\section{Deployments with CI/CD}
