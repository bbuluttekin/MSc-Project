\chapter{Conclusion}
In summary, I have implemented a functional CI/CD pipeline to predict the presence or absence of pneumonia from X-ray images.
Most of the steps I set out to achieve as part of this pipeline were successful with the exception of one. 
Originally, I have planned to implement ensemble method to increase the performance of the algorithms, but due to the limitations related to deploying multiple models into one static website, I had to abandon that step and decided to create a custom architecture instead.
Nevertheless, the decision to building custom architecture also worked as expected and model created as a product of these experiments surpassed the performance of the benchmark and transfer learning experiments and promoted to deployment.

In addition to achieving my aim, I am also proud of some other design implementations that will be useful beyond this project.
For example, the solution for resistant training module allows model training that lasts for days in a short term computational environment like Google Colaboratory.
This solution can be used in any future project and disruption to training will not affect the progress of the experimentations.
Generating a balanced dataset from imbalance data with data augmentation is another side achievement that can be applied to a wide variety of problems that suffering from data imbalance.

\section{Next Steps}
This project can be extended with the suggestions from this section to improve many aspects of the project.
I listed some steps to increase the overall quality of the implementation.

\begin{itemize}
    \item Addition hyper-parameter tuning to increase model performance.
    \item Implementing GradCAM visualization to static website deployment so the users can see the reasons for the model predictions together with the prediction.
    \item Adding more tests to increase test coverage.
    \item Bigger datasets became publicly available after I started this project. Switching to one of those dataset benefits the performance.
\end{itemize}


% Steps I have set for this project such as setting a benchmark and improving prediction performance beyond this benchmark was successful.

% "The module should demonstrate:
% - A knowledge of programming a piece of software that goes beyond a few lines of code, applying what was learnt in the eight taught modules to a concrete problem.
% - The ability to develop the design of a software solution to a concrete problem that can be identified as data analytics/data science,
% --- perform abstract thinking,
% --- exhibit abstraction skills,
% --- use a coherent development process and
% --- exhibit ability to validate and analyze the results."