\chapter{Data} \label{chap:data}
Dataset~\cite{dataset} for the project is X-ray images that collected at Guangzhou Women and Children’s Medical Center, Guangzhou from pediatric patients aged between one to five.
All X-ray images are in JPEG format and organized into folder based structure.
Main sub-structures in the dataset are train, validation and the test folders.
All main folders also broken into sub-folders that named \emph{Normal} and \emph{Pneumonia} to indicate their classification.
All X-rays have graded by expert physicians and check individually, damaged or un-readable X-rays discarded.
Below are the illustrative examples from both classes of the X-ray.

\begin{figure}[H]%
    \centering
    \subfloat[X-ray without Pnuemonia]{{\includegraphics[width=.4\textwidth]{img/chest_xray_train_NORMAL_IM-0133-0001.jpeg} }}%
    \qquad
    \subfloat[X-ray with Pnuemonia]{{\includegraphics[width=.4\textwidth]{img/chest_xray_train_PNEUMONIA_person1007_virus_1690.jpeg} }}%
    \caption{Two sample X-ray Chest images with and without pneumonia.}%
    \label{fig:sample}%
\end{figure}

For more detailed understanding of the data, I have checked the content of the each train, validation and test datasets. 
There is a significant imbalance in between Pneumonia and Normal classes for train and the test datasets.
Detailed breakdown of each dataset are given in the table \ref{table:dataset}.


\begin{table}[H]
    \centering
    \begin{tabular}{||c c c c||} 
    \hline
    Dataset Name & No of images & Pneumonia & Normal \\ [0.5ex] 
    \hline\hline
    Train & 5216 & 3875 & 1341 \\ 
    \hline
    Validation & 16 & 8 & 8 \\
    \hline
    Test & 624 & 390 & 234 \\ [1ex] 
    \hline
    \end{tabular}
    \caption{Breakdown of images for each classification folder}
    \label{table:dataset}
\end{table}



\section{Data Augmentation}
Data augmentations applied and example results of the augmentation.

\section{Limitations of the Dataset}
Issues related to validation set size. Variance of the image resolution.

\section{Data Procession}
Information about tf.data processing pipeline and advantages compare to other data feeds.

Information about data processing for scikit-learn models.
Talk about the decisions for image size and how it relates to information loss and preservation. Briefly mention the high variance of the image sizes and image resizing options and choice.
Discussing different representation of the dataset will be covered.
Talk about data module design and functionality.
\clearpage